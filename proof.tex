\documentclass[a4paper]{article}
\synctex=1
\usepackage{lmodern}
\usepackage{microtype}
\usepackage[T1]{fontenc}
\usepackage[utf8]{inputenc}
\usepackage{amsmath,amssymb,amsthm}
\usepackage{parskip}
\DeclareMathOperator{\dist}{dist}
\newcommand{\PQ}{\ensuremath{\mathsf{Q}}}

\newtheorem{lemma}{Lemma}
\newtheorem{claim}{Claim}
\newtheorem{observation}{Observation}

\usepackage[longend,vlined]{algorithm2e}
%\renewcommand\AlCapFnt{\normalfont\small}
%\renewcommand\AlCapNameSty{\normalfont\small}


\begin{document}

We do a Dijkstra-like dynamic programming.

Let $\dist(k', p, v) = \ell$ denote the fact that there is a walk from $s$ to
$v$ using $k'$ free edges, with parity $p \in \{0,1\}$ with length
$\ell$.


\begin{algorithm}[H]
  \DontPrintSemicolon
  Create a priority queue $\PQ$, and put $(0, 0, 0, s)$ into \PQ\;
  \While{$\PQ$}{
    pop $(\ell, k', p, v)$\;
    \If{$\dist(k', p, v) < \ell$}{
      continue  \tcp*[f]{discard sub-optimal vector}
    }
    $\dist(k', p, v) \gets \ell$\;
    \For{$u \in N_G(v)$}{
      \PQ.push $(\ell + w(u,v), k', \overline{p}, u)$\;
      \PQ.push $(\ell, k'+1, \overline{p}, u)$ if $k' < k$\;
    }
  }
\end{algorithm}

\begin{observation}
  Whenever a vector is pushed onto the queue, either $\ell$ increases or $k'$
  increases.
\end{observation}
\begin{claim}
  Consider the first time $\dist(k', p, v)$ is written with value $\ell$.  Then
  $\ell$ is optimal.
\end{claim}
\begin{proof}

\end{proof}

\begin{claim}
  The algorithm pops $v$ at most $2 k \deg(v)$ many times.
\end{claim}
\begin{proof}
  Notice that each entry of $v$ in the \PQ, over the entire run, has been
  pushed by a neighbor, and that for each $k$, it can be pushed either in the
  case where $k$ is decreased, or not, hence it is pushed at most $2k \deg(v)$
  times.
\end{proof}

\begin{claim}
  When the priority queue is empty, $\dist(k', p, v) = \ell$ is correct
  for every $k', p, v$.
\end{claim}
\begin{proof}
  Clearly $\dist(0, 0, s) = 0$ is correct.  Suppose for contradiction that the
  claim is false, and let $(\ell, k', p, v)$ be the lexicographically smallest
  erroneous vector, i.e., that we compute $\dist(k', p, v) = \ell' \neq \ell$,
  and that there is an $s$-$v$-walk of length $\ell$ using $k'$ free edges with
  parity $p$.

  To finish the proof, consider the walk $W$ from $s$ to $v$ of length $\ell$ that
  uses $k'$ free edges with parity $p$.
  %
  Notice that if $\ell' > \ell$ the vertex in $u \in W$ that precedes $v$ has
  either
  \begin{itemize}
  \item vector $(\ell - w(u,v), k', \overline{p}, u)$, or
  \item vector $(\ell, k' - 1, \overline{p}, u)$,
  \end{itemize}
  both of which are lexicographically smaller than $(\ell, k', p, v)$, hence
  they are correctly computed.  But then $(\ell, k', p, v)$ must be put into
  the priority queue, and hence correctly computed.

  We conclude that $\ell' < \ell$, i.e., we find a \emph{shorter} path than
  possible; we underestimate the cost of the walk.

  Let $(\ell'', k'', \overline p, u)$ be the vector popped before adding
  $(\ell, k', p, v)$ to the queue.  Notice that again, either $\ell'' = \ell$
  and $k'' < k'$, or $\ell'' < \ell$ and $k'' = k$, and by the assumption that
  $(\ell'', k'', \overline p, u)$ therefore is correct,
  $\dist(k', p, v) = \ell$, which contradicts our assumption.
\end{proof}

\begin{lemma}
  The overall running time is $O(nk \log n)$.
\end{lemma}
\begin{proof}
  From the claim above, each vertex is popped at most $2k\deg(v)$ many times,
  hence each vertex is pushed at most $2k\deg(v)$ times.  In total
  $4km = O(nk)$ many push and pop operations are performed, each running in
  time $O(\log nk) = O(\log n)$.  Since the remaining computations are all
  $O(1)$, the total running time is $O(nk \log n)$.
\end{proof}
\end{document}
